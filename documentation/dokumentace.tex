\documentclass[12pt,a4paper]{article}
\usepackage[utf8]{inputenc}
\usepackage[czech] {babel}
\usepackage[pdftex]{graphicx}
\usepackage{gensymb}
\usepackage{amsmath}
\usepackage{ulem}
\usepackage{svg}
\usepackage{mathtools}

\usepackage{newverbs}
\usepackage{listings}

% \setlength{\hoffset}{-1.8cm} 
% \setlength{\voffset}{-2cm}
% \setlength{\textheight}{24.0cm} 
%\setlength{\textwidth}{17cm}

\usepackage{fancyvrb}

\lstnewenvironment{code}[1][]%
{\noindent\minipage{\linewidth}\lstset{frameround=fttf,#1}}%
{\endminipage}%
\definecolor{codeprimary}{HTML}{3300CC}
\colorlet{keywordstyle}{codeprimary!50!black}
\lstset{
    language=SQL,
    frameround=fftf,
    breaklines=true,
    keywordstyle=\color{keywordstyle}\ttfamily,
    basicstyle=\color{codeprimary},
    numberstyle=\color{black},
    backgroundcolor=\color{white},
    frame=single,
    tabsize=4,
    breaklines=true,
    captionpos=t,
    xleftmargin=\dimexpr\fboxsep+1pt\relax,
    xrightmargin=\fboxsep,
    numbers=none,
    showstringspaces=false,
    escapeinside={\#!}{\^^M},
    belowcaptionskip=0pt,
    belowskip=0pt,
    aboveskip=0pt,
}

\makeatletter
\newcommand\ic[1][green]{%
    \@testopt{\@ic{#1}}{-#1}% Handle second optional argument
}
\def\@ic#1[#2]{%
    \Collectverb{\@@ic{#1}{#2}}%
}
\def\@@ic#1#2#3{%
    {\lstinline[basicstyle=\ttfamily\color{codeprimary},breaklines=true]|#3|}%
}
\newcommand{\icmacro}[1]{{\lstinline[basicstyle=\ttfamily\color{codeprimary},breaklines=true]|#1|}}
\makeatother

\usepackage{fancyhdr}


\usepackage[total={18cm,25cm}, top=2cm, left=2cm, right=2cm, includefoot]{geometry}

\begin{document}
	
	% UVODNI STRANA
	\begin{titlepage}
	\begin{figure}
	\centering
 	\includegraphics[scale=1.0]{obr/logo.eps}
	\end{figure}
	{\centering
	\begin{LARGE}
 	VYSOKÉ UČENÍ TECHNICKÉ V BRNĚ
	\end{LARGE}

	\vspace{20pt}
	\begin{Large}
	Fakulta informačních technologií
	\end{Large}


	\vspace{200pt}

	\begin{Large}
		Databázové systémy
	\end{Large}

	\vspace{10pt}

 	\begin{Huge}
 	Internetový obchod s pastelkami a skicáky
	\end{Huge}
	
	\vspace{10pt}

	\begin{Large}
		Dokumentace k projektu
	\end{Large}
	\\
}
	\vfill
	
	Josef Kolář \texttt{xkolar71}\\ Iva Kavánková \texttt{xkavan05} \hfill 29. dubna 2018

	\end{titlepage}

 	\tableofcontents

\pagestyle{fancy}
\lfoot{VUT FIT - IDS}
\rfoot{2017/2018}
\rhead{Josef Kolář \texttt{xkolar71}, Iva Kavánková \texttt{xkavan05}}
\lhead{Dokumentace k projektu IDS}

	\newpage
 	
	
	\section{Popis zadání}
	Cílem je vytvoření jednoduché aplikace pro internetový obchod s pastelkami a skicáky. Návštěvníci si mohou pomocí internetového rozhraní prohlížet veškerý sortiment obchodu. Pastelky mohou lišit podle typu (obyčejné, progresso, voskovky, ...) a délky, počtu pastelek v balení, atd. Skicáky se dělí podle gramáže, velikosti, počtu papírů, apod.

	Pokud má návštěvník zájem o určitý produkt/y, může si jej vybrat (vložením do nákupního košíku). U registrovaných zákazníků, kteří jsou do systému přihlášeni, zůstává informace o vybraném zboží v košíku uložena a při opětovném přihlášení znovu načtena.

	Zákazník si může zboží objednat po zadání potřebných údajů (kontakt, doprava, ...).

	Zákazníci mohou jednotlivé zboží hodnotit a psát na něj recenze. V systému jsou uloženy také základní údaje o dodavatelích pro opětovné přiobjednání dalšího zboží. Zaměstnanci mohou nahlédnout do statistik oblíbenosti a prodejnosti zboží.

	\section{Triggery}
	První trigger s názvem \ic|order_item_bi__auto_order_generator| automaticky generuje hodnoty do sloupce \ic|"order_item"."order"|, jestliže právě tento sloupec není zadaný, resp. je při vkládání \ic|NULL|. Tohoto chování lze využívat při vkládání položek do objednávky, kdy je pořadí počítání automaticky.

	Druhý trigger s názvem \ic|order_item_ad__empty_order_cleaner| zabraňuje vzniku prázdných objednávek, tzn. ve chvíli smazání objednávky, zkontroluje, zda neexistují nějaké prázdné objednávky a ty případně smaže.

	\section{Procedury}
	První procedura \ic|copy_order_items(source, target)| vezme položky zdrojové objednávky a nakopíruje je do cílové objednávky. Dochází k plytké kopii, tudíž samotné produkty zůstávají identické. Cílová objednávka musí existovat, jinak je notifikována pomocí výjimky, resp. varování uživateli.

	Druhá procedura \ic|delete_users_without_orders()| maže uživatele, pro které neexistuje v evidenci žádná objednávka.

	\section{Explain plan a index}

	Pro demostraci indexů a použití \ic|EXPLAIN PLAN| je použit dotaz typu \ic|SELECT| zjišťující, kterých dodavatelů produkty si objednali zákazníci s doručovací adresou v Brně:

	\begin{code}
SELECT
  "supplier"."name",
  count(DISTINCT "user"."id")
FROM "supplier"
  INNER JOIN "product"
  	ON "supplier"."id" = "product"."supplier_id"
  INNER JOIN "order_item"
  	ON "product"."id" = "order_item"."product_id"
  INNER JOIN "order"
  	ON "order_item"."order_id" = "order"."id"
  INNER JOIN "user"
  	ON "order"."user_id" = "user"."id"
  WHERE "user"."city" = 'Brno'
GROUP BY "supplier"."name";
	\end{code}\\

	Zde je výstup \ic|EXPLAIN PLAN| pro tento dotaz:

	{\scriptsize
	\begin{Verbatim}[commandchars=\\\{\}]
| Id  | Operation                               | Name            | Rows  | Bytes | Cost (%CPU)| Time     |
-----------------------------------------------------------------------------------------------------------
|   0 | SELECT STATEMENT                        |                 |     9 |   234 |    16  (13)| 00:00:01 |
|   1 |  HASH GROUP BY                          |                 |     9 |   234 |    16  (13)| 00:00:01 |
|   2 |   VIEW                                  | VM_NWVW_1       |     9 |   234 |    16  (13)| 00:00:01 |
|   3 |    HASH GROUP BY                        |                 |     9 |   396 |    16  (13)| 00:00:01 |
|*  4 |     HASH JOIN                           |                 |    25 |  1100 |    15   (7)| 00:00:01 |
|*  5 |      HASH JOIN                          |                 |    25 |   700 |    12   (9)| 00:00:01 |
|*  6 |       HASH JOIN                         |                 |    25 |   550 |     9  (12)| 00:00:01 |
|   7 |        MERGE JOIN                       |                 |     7 |   112 |     6  (17)| 00:00:01 |
\textcolor{codeprimary}{|*  8 |         TABLE ACCESS BY INDEX ROWID     | user            |     2 |    20 |     2   (0)| 00:00:01 |}
|   9 |          INDEX FULL SCAN                | SYS_C00535081   |    10 |       |     1   (0)| 00:00:01 |
|* 10 |         SORT JOIN                       |                 |    11 |    66 |     4  (25)| 00:00:01 |
|  11 |          TABLE ACCESS FULL              | order           |    11 |    66 |     3   (0)| 00:00:01 |
|  12 |        TABLE ACCESS FULL                | order_item      |    38 |   228 |     3   (0)| 00:00:01 |
|  13 |       TABLE ACCESS FULL                 | product         |    27 |   162 |     3   (0)| 00:00:01 |
|  14 |      TABLE ACCESS FULL                  | supplier        |    12 |   192 |     3   (0)| 00:00:01 |
	\end{Verbatim}
}
	K osmému řádek, tedy přístup do tabulky podle indexu PK, je přidána informace o filtrování právě dle města - \ic|filter("user"."city"='Brno')|. Také z něj lze vyčíst cenu tohoto kroku, tedy 2. Ta ovlivňuje cenu celého dotazu, která je stanovena na 16. Pro snížení ceny za toto filtrování můžeme vytvořit index nad tímto sloupcem, čímž zjednodušíme plánovači dotazu jeho provedení. Po přidání indexu je výstup z \ic|EXPLAIN PLAN| následující:

{\scriptsize
	\begin{Verbatim}[commandchars=\\\{\}]
|   0 | SELECT STATEMENT                        |                 |     9 |   234 |    15  (14)| 00:00:01 |
|   1 |  HASH GROUP BY                          |                 |     9 |   234 |    15  (14)| 00:00:01 |
|   2 |   VIEW                                  | VM_NWVW_1       |     9 |   234 |    15  (14)| 00:00:01 |
|   3 |    HASH GROUP BY                        |                 |     9 |   396 |    15  (14)| 00:00:01 |
|*  4 |     HASH JOIN SEMI                      |                 |    38 |  1672 |    14   (8)| 00:00:01 |
|*  5 |      HASH JOIN                          |                 |    38 |  1292 |    12   (9)| 00:00:01 |
|*  6 |       HASH JOIN                         |                 |    38 |  1064 |     9  (12)| 00:00:01 |
|   7 |        MERGE JOIN                       |                 |    27 |   594 |     6  (17)| 00:00:01 |
|   8 |         TABLE ACCESS BY INDEX ROWID     | supplier        |    12 |   192 |     2   (0)| 00:00:01 |
|   9 |          INDEX FULL SCAN                | SYS_C00535113   |    12 |       |     1   (0)| 00:00:01 |
|* 10 |         SORT JOIN                       |                 |    27 |   162 |     4  (25)| 00:00:01 |
|  11 |          TABLE ACCESS FULL              | product         |    27 |   162 |     3   (0)| 00:00:01 |
|  12 |        TABLE ACCESS FULL                | order_item      |    38 |   228 |     3   (0)| 00:00:01 |
|  13 |       TABLE ACCESS FULL                 | order           |    11 |    66 |     3   (0)| 00:00:01 |
\textcolor{codeprimary}{|  14 |      TABLE ACCESS BY INDEX ROWID BATCHED| user            |     4 |    40 |     2   (0)| 00:00:01 |}
\textcolor{codeprimary}{|* 15 |       INDEX RANGE SCAN                  | USER_CITY_INDEX |     4 |       |     1   (0)| 00:00:01 |}
	\end{Verbatim}
}
	Přidání indexu způsobilo posunutí prohledávání tabulky "user" na vyšší úroveň, čímž se i snížila celková cena dotazu na $15$. 

	\section{Přístupová práva}
	Zde došlo k předání přístupových práv uživatelem \texttt{XKAVAN05} uživateli \texttt{XKOLAR71}, který má pak možnost zasílat dotazy typu \ic|SELECT| tabulky \ic|user, order, order_item, product|.

	\section{Materializovaný pohled}
	Zde byl vytvořen materializovaný pohled \ic|v_users_with_over_average_priced_orders| na tabulkami uživatele \texttt{XKAVAN05} z účtu uživatele \texttt{XKOLAR71}:

	Následuje SQL dotaz, který ukazuje použití tohoto materializovaného pohledu:

	\vspace{15pt}

	\begin{code}
select *
from "user"
where "id" in (
	select id
	from XKOLAR71.V_USERS_WITH_OVER_AVERAGE_PRICED_ORDERS
);
	\end{code}
	
\end{document}